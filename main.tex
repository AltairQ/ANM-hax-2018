\documentclass[a4paper,twocolumn]{article}
\usepackage{geometry}
\geometry{
a4paper,
left=5mm,right=5mm,top=10mm,bottom=17mm}


\usepackage{amsmath,amssymb,textcomp}

\usepackage{polski}
\usepackage[utf8]{inputenc}
\usepackage[polish]{babel}

\pagestyle{empty}

\everymath{\displaystyle}


%\usepackage{times}
\renewcommand\familydefault{\sfdefault}
% \usepackage{tgheros}
% \usepackage[defaultmono,scale=0.15]{droidmono}


\usepackage[small]{titlesec}


\begin{document}

\fontsize{7}{4}\selectfont


\section{Analiza błędów}
\subsection{Reprezentacja liczb}
$x = smB^c$, $s$ - znak, $1 \leq m < B$ - mantysa, $c$ - cecha \\
\textbf{Def.} $t$ - długość mantysy, $u = 2^{-t-1}$ - precyzja arytmetyki \\
Błąd zaokrąglenia w górę:
$|rd(x) - x| \leq 2^{-t-1} 2^c$, \\
$\frac{|rd(x) - x|}{|x|} \leq  \frac{u}{1+u} < u$ \\
Zbiór reprezentacji arytmetyki: $X_{fl} = rd(X) = \{rd(x) : x \in X \}$
\textbf{Tw.} Jeśli $|\alpha_i| \leq u, \rho_i = \pm 1, nu < 1$, to $\prod_{i = 1}^{n} (1 + \alpha_i)^{\rho_i} = 1 + \theta_n$, gdzie $|\theta_n| \leq \frac{nu}{1 - nu}$. \\
\textbf{Tw.} Jeśli $|\alpha_i| \leq u, nu < 0.01$, to $\prod_{i = 1}^{n} (1 + \alpha_i) = 1 + \theta_n$, gdzie $|\theta_n| \leq 1.01nu$. \\
\textbf{Def.} Zadanie jest \textbf{źle uwarunkowane}, jeśli niewielkie względne zmiany danych powodują duże względne zmiany wyniku. Wielkości charakteryzujące wpływ zaburzeń danych na zaburzenie wyniku to
\textbf{wskaźniki uwarunkowania}:\\
$|\frac{f(x+h)-f(x)}{f(x)}| \approx |\frac{xf'(x)}{f(x)}||\frac{h}{x}|$\\
\textbf{Def.} Niech $\tilde{y}$ - wynik algorytmu obliczającego $f(x)$. Jeśli dla małych $\Delta x, \Delta y$ mamy $\tilde{y} + \Delta y = f(x + \Delta x)$, tzn. wynik jest lekko zaburzony dla lekko zaburzonych danych, to algorytm jest \textbf{numerycznie poprawny}. \\
Jeśli $\tilde{y} = f(x + \Delta x)$ (wynik dokładny dla lekko zaburzonych danych), to algorytm jest \textbf{numerycznie bardzo poprawny}.\\
\section{Interpolacja}

\subsection{Postacie wielomianów interpolacyjnych}
Lagrange'a:  $L_n(x) =\sum_{k=0}^ny_k
\lambda_k ,gdzie\quad \lambda_k=\frac{\prod_{j=0, j\neq k}^{n}(x-x_j)}{\prod_{j=0, j\neq k}^{n}(x_k-x_j)}$\\
Barycentryczna:
$\sigma_k = \prod^n_{j=0, j\neq k}\frac{1}{x_k-x_j}$\\
$  L_n(x) = \left\{ \begin{array}{ll}
\left. \sum_{k=0}^n\frac{\sigma_k}{x-x_k}y_k \middle/ \sum_{k=0}^n\frac{\sigma_k}{x-x_k} \right. & \textrm{gdy $x \notin \left\{ x_0,x_1,\dots,x_n \right\} $}\\
y_k & \textrm{ $ wpp. $}
\end{array} \right.
$\\
Newtona: 
$ L_n(x) = \sum_{k=0}^n b_kp_k(x)$, gdzie $p_k = \prod_{j = 0}^{k-1} (x - x_j)$, \\
$b_k = \sum_{i = 0}^k \frac{f(x_i)}{\prod_{j = 0, j\neq i}^k (x_i - x_j)}$

\subsection{Reszta wielomianu interpolacyjnego}
$f(x)-L_n(x) = \frac{1}{(n+1)!}f^{(n+1)}(\xi_x)p_{n+1}(x) = p_{n+1}(x)f[x_0, x_1, ..., x_n]$\\
$\max_{-1\leq x \leq 1} | f(x)-L_n(x) | = \frac{1}{(n+1)!}\max_{-1\leq x \leq 1} | f^{(n+1)}(x) | \max_{-1\leq x \leq 1} | p_{n+1}(x) |$
\subsection{DFT}
\texttt{input:} $[x_0, x_1, ..., x_N]$\\
\texttt{output:} $[y_0, y_1, ..., y_N]$
$y_k = \sum_{j = 0}^{N-1} x_j e^{\frac{2 \pi i j k}{N}} $ gdzie $ (0 \leqslant k < N)$
\subsection{Spline}
naturalna: $\quad s''(a)=0 \land s''(b)=0$\\
zupelna: $\quad s'(a)=f'(a) \land s'(b)=f'(b)$\\
okresowa: $\quad s'(a)=s'(b) \land s''(a)=s''(b)$\\
Fakt: $\int_a^b \left(s''(x)\right)^2 dx = \sum_{k=1}^{n-1} \left( f\left[x_k,x_{k+1}\right] - f\left[x_{k-1},x_k\right]\right)s''(x_k) $\\
$\lambda_{k}M_{k}+2M_{k}+(1-\lambda_{k})M_{k-1}=6*f[x_{k-1},x_{k},x_{k+1}]$\\ $(k=1,...,n-2)$ oraz $\lambda_{k} = h_{k}/(h_{k}+h_{k+1})$ oraz $h_{k}=x_{k} - x_{k-1}$


\section{Wielomiany Ortogonalne}

\subsection{Czebyszew I rodzaju}
$T_0(x) = 1,\;\;T_1(x) = x$\\
$T_n(x) = 2x\,T_{n-1}(x) - T_{n-2}(x)$\\
$T_{k}(x) = \cos(k\arccos x),\,x\in [-1,1]$\\
zera $T_{n+1}\;(k=0,1,\ldots n)$: \, $t_k=\cos\left(\frac{2k+1}{2n+2}\pi\right)$\\
ekstrema $T_{n}:\; u_k = \cos\left(\frac{k}{n}\pi\right)$\\
waga: $p(x) = (1-x^2)^{-1/2}$\\
$ \int_{-1}^{1} p(x)T_k(x)T_l(x) dx = 0 \qquad (k \neq l) $\\
$  \int_{-1}^{1} p(x)(T_k(x))^2 dx = \left\{ \begin{array}{ll}
\pi & \textrm{gdy $k=0$}\\
\pi/2 & \textrm{gdy $ k \geq 1 $}
\end{array} \right.$
\subsection{Legendre'a}
$P_0(x) = 1, \quad P_1(x) = x-c_1,\;$
 $   P_k(x) = (x-c_k)P_{k-1}(x) - d_k P_{k-2}(x) $\\
$
    c_k = \frac{\langle x P_{k-1}, P_{k-1} \rangle}{\langle P_{k-1}, P_{k-1} \rangle}, \quad
    d_k = \frac{\langle P_{k-1}, P_{k-1} \rangle}{\langle P_{k-2}, P_{k-2} \rangle},
$
\section{Normy}
\subsection{Wektorowe}
${\displaystyle \|\mathbf {x} \|_{p}={\bigl (}|x_{1}|^{p}+|x_{2}|^{p}+\ldots +|x_{n}|^{p}{\bigr )}^{1/p},\;\|\mathbf {x} \|_{\infty }=\max_{1 \leq j \leq n} {\bigl \{}|x_{i}| {\bigr \}}}$
Norma musi spełniać 3 warunki:\\
${\displaystyle \|x\|=0\Rightarrow x=0;\; \|\alpha x\|=|\alpha |\|x\|;\;  \|x+y\|\leqslant \|x\|+\|y\|.}$
\subsection{Macierzowe}
${\displaystyle \|\mathbf {A} \|_{1}=\max _{1 \leq j \leq n}\sum _{i}|a_{ij}|,\; \|\mathbf {A} \|_{\infty }=\max _{1\leq i\leq n}\sum _{j}|a_{ij}|,\;}$\\
${\displaystyle \|\mathbf {A} \|_{p}=\max _{\mathbf {x} \neq \mathbf {0} }{\tfrac {\|\mathbf {Ax} \|_{p}}{\|\mathbf {x} \|_{p}}}.}$\\
${\displaystyle \|\mathbf {A} \|_{E}=\sqrt{\sum_{i,j}|a_{i,j}|^2}}$ (euklidesowa/Frobeniusa)\\
${\displaystyle \|\mathbf {A} \|_{2}=\sqrt{\rho(\mathbf{A}^T\mathbf{A})}}$ (spektralna, $\rho(X) = \max |\lambda(X)|$)


\section{Aproksymacja}

\subsection{Normy, iloczyn skalarny, błąd}
$\|f\|_\infty = \sup_{x \in [a,b]}|f(x)|,\;$
$\|f\|_2 = \sqrt{\int\limits_a^b p(x)f^2(x)\,dx}$\\
$\langle f, g \rangle := \int\limits_a^b p(x)\,f(x)\,g(x)\,dx,\;\; \sqrt{\langle f, f \rangle} = \|f\|_2$\\
$E_n(f) := \inf_{W \in \Pi_n} \|f-W_n\|_\infty^T$

\subsection{Wielomian optymalny}

$w_n^* = \sum_{k=0}^n \frac{\langle f, P_k \rangle}{\langle P_k, P_k \rangle} P_k,\;$
$\|f-w_n^*\|_2 = \sqrt{\|f\|_2^2 - \sum_{k=0}^n \frac{{\langle f, P_k \rangle}^2}{\langle P_k, P_k \rangle}}$

\section{Kwadratury interpolacyjne}
\noindent $I_p(f) = \int_{a}^{b}p(x)f(x)dx \approx \sum_{k=0}^{n}A_kf(x_k)$, wiel. węzłowy $\omega(x) = (x-x_0)(x-x_1)..(x-x_n)$

\subsection{Kwadratury Newtona-Cotesa}
$h=(b-a)/n$,\quad $x_k=a+kh$\\
Wzór trapezów,  $n=1$, $A_0=A_1=h/2$, $R_1(f)=-\frac{h^3}{12}f''(\xi)$\\
Wzór Simpsona, $n=2$,$A_0=A_2=h/3$, $A_1=4h/3$, $R_2(f)=-\frac{h^5}{90}f^{(4)}(\xi)$\\
Zł. wzór trapezów: $T_n=h\sum_{k=0}^{n}$ $''$ $f(x_k)$,
$R_n^T(f)=-\frac{h^3n}{12}f''(\xi)$\\
Zł. wzór Simpsona, $n=2m$,\quad $S_n(f)=(4T_n-T_m)/3$\\
$R_n^S(f)=-(b-a)\frac{h^4}{180}f^{(4)}(\xi)$\\
Metoda Romberga: $h_k=(b-a)/2^k$\\
$x_i^(k)=a+ih_k,(i=0,1,\dots,2^k)$,\quad $T_{0k}=T_{2^k}(f)$\\
$T_{mk}=\frac{4^mT_{m-1,k+1} - T{m-1,k}}{4^m-1}$\\
Reszta Newtona-Cotesa:\\
\noindent$  I_p(f) - Q_n(f) = \left\{ \begin{array}{ll}
\left.\frac{f^{(n+1)}(\xi)}{(n+1)!}\int_{a}^{b} \omega(x) dx
\right. & \textrm{gdy $ n = 1,3,... $}\\
\frac{f^{(n+2)}(\xi)}{(n+2)!}\int_{a}^{b} x\omega(x) dx & \textrm{gdy $ n = 2,4,... $}
\end{array} \right.
$
gdzie $\xi \in (a, b).$
\subsection{Kwadratury Gaussa}
\noindent$x_k$ - zera wiel. orto. $P_{n+1}(x), c_k $ wsp. wiod. $P_k$\\
$\lambda_k(x) = \omega(x) / [\omega'(x_k)(x-x_k)]. $ \\
$0 < A_k=\int_{a}^{b}p(x)\lambda_k(x) dx = \frac{c_{n+1}}{c_n} \frac{||P_n||^2}{P_{n+1}^{'}(x_k)P_n(x_k)}$.  \\
$R_n(f)=\frac{f^{(n+2)}(\xi)}{(2n+2)!c_{n+1}^{2} }\int_a^bp(x)[P_{n+1}(x)]^2 dx $
\subsection{Gauss-Czebyszew}
\noindent
%$x_k=t_k=\cos(\frac{2k+1}{2n+2}\pi)$ \\
$\int_{-1}^1p(x)f(x)dx \approx \int_{-1}^1p(x)I_n(x)dx$ \\
$ I_n(x)=\sum_{i=0}^{n}  \prime \alpha_iT_i(x), \alpha_i=\frac{2}{n+1}\sum_{i=0}^nf(t_k)T_i(t_k)$\\
$Q_n^{QC}(f)=\sum_{k=0}^{n}A_kf(t_k), A_k=\frac{\pi}{n+1}$
\subsection{Gauss-Lobatto}
\noindent
$x_k=u_k=\cos\left(\frac{k}{n}\pi\right)$ \\
$\int_{-1}^1p(x)f(x)dx \approx \int_{-1}^1p(x)J_n(x)dx$ \\
$ J_n(x)=\sum_{j=0}^{n}  $ $''$ $ \beta_jT_j(x), \beta_j=\frac{2}{n}\sum_{k=0}^n $ $''$ $  (u_k)T_j(u_k)$\\
$Q_n^{L}(f)=\sum_{k=0}^{n} $ $''$ $ A_kf(u_k), A_k=\frac{\pi}{n}.$



\section{Algebra numeryczna}
\subsection{Metody iteracyjne}
${x}^{(k+1)} = B_{\tau} {x}^{(k)} + {c} \hspace{1em} (k \geq 0)$\\
\textbf{Tw.} Metoda iter. jest zbieżna dla dowolnego $x^{(0)}$, jeśli $\rho(B)<1$,\\
gdzie $\rho(B):=\max_{1 \leq i \leq n} (\lambda_i)$ to promień spektralny macierzy B.\\
$B_{\tau} = I - \tau A, \hspace{1em} c = \tau b \hspace{1em} \mbox{(Metoda Richardsona)}$\\
\textbf{Tw.} Załóżmy, że wszystkie $\lambda_i$ spełniają: $0 < \alpha \leq \lambda_i \leq \beta$.\\
Wtedy metoda Richardsona jest zbieżna dla $0 < \tau < \frac{2}{\beta}$.\\
$B_J = -D^{-1}(L+U) \hspace{1em} \mbox{(Metoda Jacobiego)}$\\
\textbf{Tw.} Jeśli A jest macierzą ze ściśle dominującą przekątną, tj.\\
$|a_{ii}|>\sum\limits_{\begin{smallmatrix}j=1\\j\neq i\end{smallmatrix}}^{n}|a_{ij}|$, to wtedy $\|B_J\|_\infty < 1$ i $\|B_S\|_\infty < 1$.\\
$B_S = -(D+L)^{-1}U \hspace{1em} \mbox{(Metoda Gaussa-Seidela)}$\\
\textbf{Tw.} Jeśli A jest symetryczna i dodatnio określona, to $\|B_S\|_\infty < 1$.\\
$B_{\omega} = (I - \omega M)^{-1} (\omega N + (1 - \omega)I), \hspace{1em} M = -D^{-1}L, N = -D^{-1} U$

\section{Równania nieliniowe}
Metoda Newtona: $X_{n+1}  = X_n - \frac{f(X_n)}{f'(X_n)}$\\
Metoda siecznych: $X_{n+1}  = X_n - \frac{X_n - X_{n-1}}{f(X_n)- f(X_{n-1})}$\\
Regula falsi - metoda siecznych z zachowaniem warunku $f(X_n)f(X_{n-1})<0$
Metoda Newtona dla r-krotnego pierwiastka: $X_{n+1}  = X_n - r\frac{f(X_n)}{f'(X_n)}$\\
Algortym adaptacyjny: $X_{n+1}  = X_n - r_n\frac{f(X_n)}{f'(X_n)}$, $r_n = \frac{X_{n-1} - X_{n-2}}{2X_{n-1} - X_n - X_{n-2}}$\\
Metoda Newtona dla wielu zmiennych: $J*h = -F$,  $X_{n+1} = X_n + h_n$\\
Metoda Laguerra(zbieżna sześciennie do pojedynczych pierwiastków): \\ $Z_{k+1} = Z_k - \frac{n\omega (Z_k)}{\omega '(Z_k) +/- \sqrt[2]{H(Z_k)}}$,\\ $H(x) = (n-1)[(n-1) \omega '^2(x) - n\omega (x) \omega ''(x)]$\\
Metoda Steffensena: $C_{n+1} = C_n - \frac{f^2(C_n)}{f(C_n + f(C_n)) - f(C_n)}$

\section{Teoretycznie przydatne wzory}

Nierówność Cauchy'ego-Schwarza $ \Big(\sum_{k=1}^{n} a_k b_k\Big)^2 \leq \Big(\sum_{k=1}^{n} a_k^2\Big)\Big(\sum_{k=1}^{n} b_k^2\Big) $ \\
Inaczej: $\langle v, w\rangle \leq \|v\|_2\cdot \|w\|_2$\\
Bairstowa: $p(z)=a_nz^n+\dots+a_0$
zwiazek: $b_k=a_k+ub_{k+1}+vb_{k+2}$\\
Niech ciąg $a_k$ będzie zbieżny do g. Jeśli istnieją takie liczby rzeczywiste $p$ i $C$ $(C > 0)$, że
$\lim_{n \rightarrow \inf}{\frac{|{a_{n+1}-g}|}{|{a_n - g}|^p}} = C, $
to $p$ jest wykładnikiem zbieżności ciągu, a $C$ – stałą
asymptotyczną błędu. Dla $p = 1$ oraz $0 < C < 1$ zbieżność jest
liniowa, dla $p = 2$ – kwadratowa, dla $p = 3$ – sześcienna.
gdy $p = 1$ i $C = 1$ - zbieżność podliniowa, jeśli $p = 1$, $C = 0$ - nadliniowa.\\
${\displaystyle \sin x\pm \sin y=2\sin {\frac {x\pm y}{2}}\cdot \cos {\frac {x\mp y}{2}}} $\\
$\cos x+\cos y=2\cos {\frac  {x+y}2}\cdot \cos {\frac  {x-y}2}$\\
${\displaystyle \sin(x\pm y)=\sin x\cdot \cos y\pm \cos x\cdot \sin y}$\\
${\displaystyle \cos(x\pm y)=\cos x\cdot \cos y\mp \sin x\cdot \sin y} $\\
$\cos x-\cos y=-2\sin {\frac  {x+y}2}\cdot \sin {\frac  {x-y}2}$\\
Wzór Taylora : $f(x) = \sum_{k=0}^n \frac{(x-a)^k}{k!}f^{(k)}(a) + R_n(x, a)$, $\lim_{x->a} \frac{R_n(x, a)}{(x-a)^n} = 0$
Wzór Taylora dla dwóch zmiennych: $f(a + h) = f(a) + h_1 \frac{df}{dx_1}(x_1, x_2) + h_2 \frac{df}{dx_2}(x_1, x_2)... $\\


\end{document}
